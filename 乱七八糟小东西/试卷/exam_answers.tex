\documentclass[a4paper,12pt]{article}
\usepackage{amsmath,amssymb,amsthm}
\usepackage{geometry}
\usepackage{fancyhdr}
\usepackage{graphicx}
\usepackage{multicol}
\usepackage{enumitem}
\usepackage[UTF8]{ctex}

\geometry{top=1cm,bottom=1cm,left=1cm,right=1cm}

\pagestyle{fancy}
\fancyhf{}
\renewcommand{\headrulewidth}{0pt}

\begin{document}

\begin{center}
\Large\textbf{统计学 2024-2025 第一学期 单元测试答案}

\vspace{0.5cm}
\normalsize 2024.10
\end{center}

\vspace{0.5cm}

\begin{enumerate}[leftmargin=*]

    \item (10分) 证明:
    \[
    \cos\frac{2\pi}{n} + 2\cos\frac{4\pi}{n} + \cdots + (n-1)\cos\frac{2(n-1)\pi}{n} = -\frac{n}{2}
    \]
    \[
    \sin\frac{2\pi}{n} + 2\sin\frac{4\pi}{n} + \cdots + (n-1)\sin\frac{2(n-1)\pi}{n} = -\frac{n}{2}\cot\frac{\pi}{n}
    \]
    
    答案:令 $\omega = e^{2\pi i/n}$,则:
    $S = \omega + 2\omega^2 + \cdots + (n-1)\omega^{n-1}$
    $\omega S = \omega^2 + 2\omega^3 + \cdots + (n-1)\omega^n = S - (n-1)\omega^n + n\omega - \omega$
    $(\omega - 1)S = n\omega - \omega - (n-1) = (n-1)(\omega - 1)$
    $S = n-1 + \frac{n-1}{\omega-1} = \frac{n}{2} - \frac{n}{2}\cot\frac{\pi}{n} + i\frac{n}{2}$
    因此,$\text{Re}(S) = -\frac{n}{2}$,$\text{Im}(S) = -\frac{n}{2}\cot\frac{\pi}{n}$
    
    \item (10分) 用数学归纳法证明范德蒙行列式:
    \[
    D_n = \begin{vmatrix}
    1 & 1 & \cdots & 1 \\
    x_1 & x_2 & \cdots & x_n \\
    x_1^2 & x_2^2 & \cdots & x_n^2 \\
    \vdots & \vdots & \ddots & \vdots \\
    x_1^{n-1} & x_2^{n-1} & \cdots & x_n^{n-1}
    \end{vmatrix} = \prod_{1\leq i < j \leq n}(x_j - x_i)
    \]
    
    答案:用数学归纳法证明。
    当 $n=1$ 时,$D_1 = 1$,结论成立。
    假设 $n-1$ 阶范德蒙行列式成立,考虑 $n$ 阶范德蒙行列式:
    将第 $j$ 列 $(j=2,\ldots,n)$ 减去第 1 列,得到:
    \[
    D_n = \begin{vmatrix}
    1 & 0 & \cdots & 0 \\
    x_1 & x_2-x_1 & \cdots & x_n-x_1 \\
    x_1^2 & x_2^2-x_1^2 & \cdots & x_n^2-x_1^2 \\
    \vdots & \vdots & \ddots & \vdots \\
    x_1^{n-1} & x_2^{n-1}-x_1^{n-1} & \cdots & x_n^{n-1}-x_1^{n-1}
    \end{vmatrix}
    \]
    提取公因子,得到:
    \[
    D_n = \prod_{j=2}^n (x_j-x_1) \cdot D_{n-1}(x_2,\ldots,x_n)
    \]
    根据归纳假设,
    \[
    D_n = \prod_{j=2}^n (x_j-x_1) \cdot \prod_{2\leq i < j \leq n}(x_j - x_i) = \prod_{1\leq i < j \leq n}(x_j - x_i)
    \]
    因此,结论对 $n$ 阶范德蒙行列式也成立。
    
    \item (15分) 设$a_1, a_2, \ldots, a_n$是数域$P$中互不相同的数,$b_1, b_2, \ldots, b_n$是数域$P$中任意给定的数,用克拉默法则证明:存在唯一的数域$P$中的多项式
    \[
    f(x) = c_0x^{n-1} + c_1x^{n-2} + c_2x^{n-3} + \cdots + c_{n-1}
    \]
    使得
    \[
    f(a_1) = b_1, f(a_2) = b_2, \ldots, f(a_n) = b_n
    \]
    
    答案:构造线性方程组:
    \[
    \begin{cases}
    c_0a_1^{n-1} + c_1a_1^{n-2} + \cdots + c_{n-1} = b_1 \\
    c_0a_2^{n-1} + c_1a_2^{n-2} + \cdots + c_{n-1} = b_2 \\
    \vdots \\
    c_0a_n^{n-1} + c_1a_n^{n-2} + \cdots + c_{n-1} = b_n
    \end{cases}
    \]
    系数矩阵为范德蒙矩阵,其行列式不为零(因为$a_i$互不相同)。根据克拉默法则,该方程组有唯一解,即存在唯一的多项式$f(x)$满足给定条件。
    
    \item (10分) 计算行列式:
    \[
    D_n = \begin{vmatrix}
    \cos\theta & -\sin\theta & 0 & \cdots & 0 \\
    \sin\theta & \cos\theta & 0 & \cdots & 0 \\
    0 & \sin\theta & \cos\theta & \cdots & 0 \\
    \vdots & \vdots & \vdots & \ddots & \vdots \\
    0 & 0 & 0 & \cdots & \cos\theta
    \end{vmatrix}
    \]
    
    答案:$D_n = (\cos\theta)^n$
    证明:用数学归纳法。当$n=1$时,$D_1 = \cos\theta$,结论成立。
    假设$n-1$阶行列式结论成立,考虑$n$阶行列式:
    沿第一行展开,得到:
    $D_n = \cos\theta \cdot D_{n-1} + \sin\theta \cdot \begin{vmatrix}
    \sin\theta & 0 & \cdots & 0 \\
    \cos\theta & 0 & \cdots & 0 \\
    \vdots & \vdots & \ddots & \vdots \\
    0 & 0 & \cdots & \cos\theta
    \end{vmatrix}$
    第二项行列式为0,因此$D_n = \cos\theta \cdot D_{n-1} = \cos\theta \cdot (\cos\theta)^{n-1} = (\cos\theta)^n$
    
    \item (10分) 计算行列式:
    \[
    D_n = \begin{vmatrix}
    1 & 2 & 3 & \cdots & n \\
    2 & 3 & 4 & \cdots & 1 \\
    3 & 4 & 5 & \cdots & 2 \\
    \vdots & \vdots & \vdots & \ddots & \vdots \\
    n & 1 & 2 & \cdots & n-1
    \end{vmatrix}
    \]
    
    答案:$D_n = (-1)^{n-1}(n-1)$
    证明:对行列式进行初等变换:
    第2行减去第1行,第3行减去第2行,...,第n行减去第n-1行,得到:
    \[
    D_n = \begin{vmatrix}
    1 & 2 & 3 & \cdots & n \\
    1 & 1 & 1 & \cdots & 1-n \\
    1 & 1 & 1 & \cdots & 1 \\
    \vdots & \vdots & \vdots & \ddots & \vdots \\
    n-1 & -1 & -1 & \cdots & 1
    \end{vmatrix}
    \]
    第2列减去第1列,第3列减去第2列,...,第n列减去第n-1列,得到:
    \[
    D_n = \begin{vmatrix}
    1 & 1 & 1 & \cdots & n-1 \\
    1 & 0 & 0 & \cdots & -n \\
    1 & 0 & 0 & \cdots & 0 \\
    \vdots & \vdots & \vdots & \ddots & \vdots \\
    n-1 & -n & 0 & \cdots & 2
    \end{vmatrix}
    \]
    沿第一列展开,得到:
    $D_n = 1 \cdot (-1)^{1+1} \cdot (-n) \cdot (-1)^{n-2} + (n-1) \cdot (-1)^{n+1} \cdot (-n) = (-1)^{n-1}(n-1)$
    
    \item (15分) 计算行列式:
    \[
    D_n = \begin{vmatrix}
    0 & a & a & \cdots & a \\
    b & 0 & a & \cdots & a \\
    b & b & 0 & \cdots & a \\
    \vdots & \vdots & \vdots & \ddots & \vdots \\
    b & b & b & \cdots & 0
    \end{vmatrix}
    \]
    
    答案:$D_n = (-1)^{n-1}(n-1)ab^{n-1} - (-a)^n - (-b)^n$
    证明:用数学归纳法。当$n=2$时,$D_2 = -a^2-b^2$,结论成立。
    假设$n-1$阶行列式结论成立,考虑$n$阶行列式:
    沿第一行展开,得到:
    $D_n = a(D_{n-1} + b^{n-1}) - (-b)^n$
    根据归纳假设,
    $D_n = a[(-1)^{n-2}(n-2)ab^{n-2} - (-a)^{n-1} - (-b)^{n-1} + b^{n-1}] - (-b)^n$
    $= (-1)^{n-1}(n-1)ab^{n-1} - (-a)^n - (-b)^n$
    因此,结论对$n$阶行列式也成立。
    
    \item (10分) 计算行列式:
    \[
    \det\begin{pmatrix}
    \frac{1-a_1^nb_1^n}{1-a_1b_1} & \cdots & \frac{1-a_1^nb_n^n}{1-a_1b_n} \\
    \vdots & \ddots & \vdots \\
    \frac{1-a_n^nb_1^n}{1-a_nb_1} & \cdots & \frac{1-a_n^nb_n^n}{1-a_nb_n}
    \end{pmatrix}
    \]
    
    答案:该行列式的值为 $\prod_{i=1}^n (a_i^n - b_i^n)$。
    
    证明:
    令 $c_{ij} = \frac{1-a_i^nb_j^n}{1-a_ib_j}$,则 $c_{ij} = 1 + a_ib_j + (a_ib_j)^2 + \cdots + (a_ib_j)^{n-1}$。
    对行列式进行初等变换:
    第 $i$ 行乘以 $(1-a_ib_i)$,得到:
    \[
    \det\begin{pmatrix}
    1-a_1^nb_1^n & \cdots & (1-a_1b_1)\frac{1-a_1^nb_n^n}{1-a_1b_n} \\
    \vdots & \ddots & \vdots \\
    (1-a_nb_1)\frac{1-a_n^nb_1^n}{1-a_nb_1} & \cdots & 1-a_n^nb_n^n
    \end{pmatrix}
    \]
    第 $j$ 列减去第 $i$ 列 $(j \neq i)$,得到:
    \[
    \det\begin{pmatrix}
    1-a_1^nb_1^n & \cdots & 0 \\
    \vdots & \ddots & \vdots \\
    0 & \cdots & 1-a_n^nb_n^n
    \end{pmatrix}
    \]
    因此,行列式的值为 $\prod_{i=1}^n (1-a_i^nb_i^n)$。
    将每一项分解,得到 $\prod_{i=1}^n (a_i^n - b_i^n)$。
    
    \item (10分) 计算行列式 $(n \geq 2)$:
    \[
    \det\begin{pmatrix}
    x_1 & a_1b_2 & \cdots & a_1b_n \\
    a_2b_1 & x_2 & \cdots & a_2b_n \\
    \vdots & \vdots & \ddots & \vdots \\
    a_nb_1 & a_nb_2 & \cdots & x_n
    \end{pmatrix}
    \]
    
    答案:该行列式的值为 $\prod_{i=1}^n x_i - \prod_{i=1}^n a_i \cdot \prod_{i=1}^n b_i$。
    
    证明:
    令 $D_n$ 表示 $n$ 阶行列式的值。使用数学归纳法:
    
    当 $n=2$ 时,$D_2 = x_1x_2 - a_1b_1a_2b_2$,结论成立。
    
    假设 $n-1$ 阶行列式结论成立,考虑 $n$ 阶行列式:
    将第一列展开,得到:
    $D_n = x_1D_{n-1} - \sum_{i=2}^n (-1)^{i+1} a_ib_1 M_{i1}$
    其中 $M_{i1}$ 是余子式,可以写成 $(n-1)$ 阶行列式的形式:
    $M_{i1} = (-1)^{i+1} a_1 \prod_{j=2,j\neq i}^n b_j \cdot D_{n-2}$
    
    代入得到:
    $D_n = x_1D_{n-1} - a_1b_1 \prod_{j=2}^n a_j \cdot \prod_{j=2}^n b_j$
    
    根据归纳假设,有:
    $D_{n-1} = \prod_{i=2}^n x_i - \prod_{i=2}^n a_i \cdot \prod_{i=2}^n b_i$
    
    代入得到:
    $D_n = x_1(\prod_{i=2}^n x_i - \prod_{i=2}^n a_i \cdot \prod_{i=2}^n b_i) - a_1b_1 \prod_{j=2}^n a_j \cdot \prod_{j=2}^n b_j$
    $    = \prod_{i=1}^n x_i - \prod_{i=1}^n a_i \cdot \prod_{i=1}^n b_i$
    
    因此,结论对 $n$ 阶行列式也成立。
    
    \item (10分) 计算行列式 $(n \geq 2)$:
    \[
    \det\begin{pmatrix}
    \frac{1}{x_1+y_1} & \frac{1}{x_1+y_2} & \cdots & \frac{1}{x_1+y_n} \\
    \frac{1}{x_2+y_1} & \frac{1}{x_2+y_2} & \cdots & \frac{1}{x_2+y_n} \\
    \vdots & \vdots & \ddots & \vdots \\
    \frac{1}{x_n+y_1} & \frac{1}{x_n+y_2} & \cdots & \frac{1}{x_n+y_n}
    \end{pmatrix}
    \]
    
    答案:该行列式的值为 $\frac{\prod_{1\leq i<j\leq n}(x_j-x_i)(y_j-y_i)}{\prod_{i,j=1}^n(x_i+y_j)}$。
    
    证明:
    令 $a_{ij} = \frac{1}{x_i+y_j}$,则行列式可以写成 $\det(a_{ij})_{n\times n}$。
    
    考虑 $\det(a_{ij}(x_i+y_j))_{n\times n} = \det(a_{ij})_{n\times n} \cdot \prod_{i,j=1}^n(x_i+y_j)$。
    
    对 $\det(a_{ij}(x_i+y_j))_{n\times n}$ 进行行列式变换:
    第 $i$ 行减去第 1 行 $(i=2,\ldots,n)$,得到:
    \[
    \det\begin{pmatrix}
    1 & 1 & \cdots & 1 \\
    \frac{x_1-x_2}{(x_1+y_1)(x_2+y_1)} & \frac{x_1-x_2}{(x_1+y_2)(x_2+y_2)} & \cdots & \frac{x_1-x_2}{(x_1+y_n)(x_2+y_n)} \\
    \vdots & \vdots & \ddots & \vdots \\
    \frac{x_1-x_n}{(x_1+y_1)(x_n+y_1)} & \frac{x_1-x_n}{(x_1+y_2)(x_n+y_2)} & \cdots & \frac{x_1-x_n}{(x_1+y_n)(x_n+y_n)}
    \end{pmatrix}
    \]
    
    第 $j$ 列减去第 1 列 $(j=2,\ldots,n)$,得到:
    \[
    \det\begin{pmatrix}
    1 & 0 & \cdots & 0 \\
    \frac{x_1-x_2}{(x_1+y_1)(x_2+y_1)} & \frac{(x_1-x_2)(y_1-y_2)}{(x_1+y_1)(x_1+y_2)(x_2+y_1)(x_2+y_2)} & \cdots & \frac{(x_1-x_2)(y_1-y_n)}{(x_1+y_1)(x_1+y_n)(x_2+y_1)(x_2+y_n)} \\
    \vdots & \vdots & \ddots & \vdots \\
    \frac{x_1-x_n}{(x_1+y_1)(x_n+y_1)} & \frac{(x_1-x_n)(y_1-y_2)}{(x_1+y_1)(x_1+y_2)(x_n+y_1)(x_n+y_2)} & \cdots & \frac{(x_1-x_n)(y_1-y_n)}{(x_1+y_1)(x_1+y_n)(x_n+y_1)(x_n+y_n)}
    \end{pmatrix}
    \]
    
    继续进行类似的变换,最终得到:
    \[
    \det(a_{ij}(x_i+y_j))_{n\times n} = \prod_{1\leq i<j\leq n}(x_j-x_i)(y_j-y_i)
    \]
    
    因此,
    \[
    \det(a_{ij})_{n\times n} = \frac{\prod_{1\leq i<j\leq n}(x_j-x_i)(y_j-y_i)}{\prod_{i,j=1}^n(x_i+y_j)}
    \]

\item (10��) ��������ʽ��
\[
\det\begin{pmatrix}
\frac{1-a_1^nb_1^n}{1-a_1b_1} & \cdots & \frac{1-a_1^nb_n^n}{1-a_1b_n} \\
\vdots & \ddots & \vdots \\
\frac{1-a_n^nb_1^n}{1-a_nb_1} & \cdots & \frac{1-a_n^nb_n^n}{1-a_nb_n}
\end{pmatrix}
\]

答案:该行列式的值为 $\prod_{i=1}^n (a_i^n - b_i^n)$。

证明:
令 $c_{ij} = \frac{1-a_i^nb_j^n}{1-a_ib_j}$,则 $c_{ij} = 1 + a_ib_j + (a_ib_j)^2 + \cdots + (a_ib_j)^{n-1}$。
对行列式进行初等变换:
第 $i$ 行乘以 $(1-a_ib_i)$,得到:
\[
\det\begin{pmatrix}
1-a_1^nb_1^n & \cdots & (1-a_1b_1)\frac{1-a_1^nb_n^n}{1-a_1b_n} \\
\vdots & \ddots & \vdots \\
(1-a_nb_1)\frac{1-a_n^nb_1^n}{1-a_nb_1} & \cdots & 1-a_n^nb_n^n
\end{pmatrix}
\]
第 $j$ 列减去第 $i$ 列 $(j \neq i)$,得到:
\[
\det\begin{pmatrix}
1-a_1^nb_1^n & \cdots & 0 \\
\vdots & \ddots & \vdots \\
0 & \cdots & 1-a_n^nb_n^n
\end{pmatrix}
\]
因此,行列式的值为 $\prod_{i=1}^n (1-a_i^nb_i^n)$。
将每一项分解,得到 $\prod_{i=1}^n (a_i^n - b_i^n)$。

\item (10分) 计算行列式 $(n \geq 2)$:
\[
\det\begin{pmatrix}
x_1 & a_1b_2 & \cdots & a_1b_n \\
a_2b_1 & x_2 & \cdots & a_2b_n \\
\vdots & \vdots & \ddots & \vdots \\
a_nb_1 & a_nb_2 & \cdots & x_n
\end{pmatrix}
\]

答案:该行列式的值为 $\prod_{i=1}^n x_i - \prod_{i=1}^n a_i \cdot \prod_{i=1}^n b_i$。

证明:
令 $D_n$ 表示 $n$ 阶行列式的值。使用数学归纳法:

当 $n=2$ 时,$D_2 = x_1x_2 - a_1b_1a_2b_2$,结论成立。

假设 $n-1$ 阶行列式结论成立,考虑 $n$ 阶行列式:
将第一列展开,得到:
$D_n = x_1D_{n-1} - \sum_{i=2}^n (-1)^{i+1} a_ib_1 M_{i1}$
其中 $M_{i1}$ 是余子式,可以写成 $(n-1)$ 阶行列式的形式:
$M_{i1} = (-1)^{i+1} a_1 \prod_{j=2,j\neq i}^n b_j \cdot D_{n-2}$

代入得到:
$D_n = x_1D_{n-1} - a_1b_1 \prod_{j=2}^n a_j \cdot \prod_{j=2}^n b_j$

根据归纳假设,有:
$D_{n-1} = \prod_{i=2}^n x_i - \prod_{i=2}^n a_i \cdot \prod_{i=2}^n b_i$

代入得到:
$D_n = x_1(\prod_{i=2}^n x_i - \prod_{i=2}^n a_i \cdot \prod_{i=2}^n b_i) - a_1b_1 \prod_{j=2}^n a_j \cdot \prod_{j=2}^n b_j$
$    = \prod_{i=1}^n x_i - \prod_{i=1}^n a_i \cdot \prod_{i=1}^n b_i$

因此,结论对 $n$ 阶行列式也成立。

\item (10分) 计算行列式 $(n \geq 2)$:
\[
\det\begin{pmatrix}
\frac{1}{x_1+y_1} & \frac{1}{x_1+y_2} & \cdots & \frac{1}{x_1+y_n} \\
\frac{1}{x_2+y_1} & \frac{1}{x_2+y_2} & \cdots & \frac{1}{x_2+y_n} \\
\vdots & \vdots & \ddots & \vdots \\
\frac{1}{x_n+y_1} & \frac{1}{x_n+y_2} & \cdots & \frac{1}{x_n+y_n}
\end{pmatrix}
\]

答案:该行列式的值为 $\frac{\prod_{1\leq i<j\leq n}(x_j-x_i)(y_j-y_i)}{\prod_{i,j=1}^n(x_i+y_j)}$。

证明:
令 $a_{ij} = \frac{1}{x_i+y_j}$,则行列式可以写成 $\det(a_{ij})_{n\times n}$。

考虑 $\det(a_{ij}(x_i+y_j))_{n\times n} = \det(a_{ij})_{n\times n} \cdot \prod_{i,j=1}^n(x_i+y_j)$。

对 $\det(a_{ij}(x_i+y_j))_{n\times n}$ 进行行列式变换:
第 $i$ 行减去第 1 行 $(i=2,\ldots,n)$,得到:
\[
\det\begin{pmatrix}
1 & 1 & \cdots & 1 \\
\frac{x_1-x_2}{(x_1+y_1)(x_2+y_1)} & \frac{x_1-x_2}{(x_1+y_2)(x_2+y_2)} & \cdots & \frac{x_1-x_2}{(x_1+y_n)(x_2+y_n)} \\
\vdots & \vdots & \ddots & \vdots \\
\frac{x_1-x_n}{(x_1+y_1)(x_n+y_1)} & \frac{x_1-x_n}{(x_1+y_2)(x_n+y_2)} & \cdots & \frac{x_1-x_n}{(x_1+y_n)(x_n+y_n)}
\end{pmatrix}
\]

第 $j$ 列减去第 1 列 $(j=2,\ldots,n)$,得到:
\[
\det\begin{pmatrix}
1 & 0 & \cdots & 0 \\
\frac{x_1-x_2}{(x_1+y_1)(x_2+y_1)} & \frac{(x_1-x_2)(y_1-y_2)}{(x_1+y_1)(x_1+y_2)(x_2+y_1)(x_2+y_2)} & \cdots & \frac{(x_1-x_2)(y_1-y_n)}{(x_1+y_1)(x_1+y_n)(x_2+y_1)(x_2+y_n)} \\
\vdots & \vdots & \ddots & \vdots \\
\frac{x_1-x_n}{(x_1+y_1)(x_n+y_1)} & \frac{(x_1-x_n)(y_1-y_2)}{(x_1+y_1)(x_1+y_2)(x_n+y_1)(x_n+y_2)} & \cdots & \frac{(x_1-x_n)(y_1-y_n)}{(x_1+y_1)(x_1+y_n)(x_n+y_1)(x_n+y_n)}
\end{pmatrix}
\]

继续进行类似的变换,最终得到:
\[
\det(a_{ij}(x_i+y_j))_{n\times n} = \prod_{1\leq i<j\leq n}(x_j-x_i)(y_j-y_i)
\]

\end{enumerate}

\end{document}
