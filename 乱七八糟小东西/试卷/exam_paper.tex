\documentclass[a4paper,12pt]{article}
\usepackage{amsmath,amssymb,amsthm}
\usepackage{geometry}
\usepackage{fancyhdr}
\usepackage{graphicx}
\usepackage{multicol}
\usepackage{enumitem}
\usepackage[UTF8]{ctex}  % 添加此行以支持中文

\geometry{top=1cm,bottom=1cm,left=1cm,right=1cm}

\pagestyle{fancy}
\fancyhf{}
\renewcommand{\headrulewidth}{0pt}

\begin{document}

\begin{center}
\Large\textbf{统计学 2024-2025 第一学期 单元测试}

\vspace{0.5cm}
\normalsize 2024.10
\end{center}

\vspace{0.5cm}

\begin{enumerate}[leftmargin=*]

\item (10分) 证明:
\[
\cos\frac{2\pi}{n} + 2\cos\frac{4\pi}{n} + \cdots + (n-1)\cos\frac{2(n-1)\pi}{n} = -\frac{n}{2}
\]
\[
\sin\frac{2\pi}{n} + 2\sin\frac{4\pi}{n} + \cdots + (n-1)\sin\frac{2(n-1)\pi}{n} = -\frac{n}{2}\cos\frac{\pi}{n}
\]

\item (10分) 用数学归纳法证明范德蒙行列式:
\[
D_n = \begin{vmatrix}
1 & 1 & \cdots & 1 \\
x_1 & x_2 & \cdots & x_n \\
x_1^2 & x_2^2 & \cdots & x_n^2 \\
\vdots & \vdots & \ddots & \vdots \\
x_1^n & x_2^n & \cdots & x_n^n
\end{vmatrix} = \prod_{1\leq i < j \leq n}(x_j - x_i)
\]

\item (15分) 设$a_1, a_2, \ldots, a_n$是数域$P$中互不相同的数,$b_1, b_2, \ldots, b_n$是数域$P$中任意给定的数,用克拉默法则证明:存在唯一的数域$P$中的多项式
\[
f(x) = c_0x^{n-1} + c_1x^{n-2} + c_2x^{n-3} + \cdots + c_{n-1}
\]
使得
\[
f(a_1) = b_1, f(a_2) = b_2, \ldots, f(a_n) = b_n
\]

\item (10分) 计算行列式:
\[
D_n = \begin{vmatrix}
\cos\theta & -\sin\theta & 0 & \cdots & 0 \\
\sin\theta & \cos\theta & 0 & \cdots & 0 \\
0 & \sin\theta & \cos\theta & \cdots & 0 \\
\vdots & \vdots & \vdots & \ddots & \vdots \\
0 & 0 & 0 & \cdots & \cos\theta
\end{vmatrix}
\]
其中矩阵为$n$阶行列式。

\item (10分) 计算行列式:
\[
D_n = \begin{vmatrix}
\frac{1-a_1^nb_1^n}{1-a_1b_1} & \cdots & \frac{1-a_1^nb_n^n}{1-a_1b_n} \\
\vdots & \ddots & \vdots \\
\frac{1-a_n^nb_1^n}{1-a_nb_1} & \cdots & \frac{1-a_n^nb_n^n}{1-a_nb_n}
\end{vmatrix}
\]

\item (10分) 计算行列式 $(n \geq 2)$:
\[
D_n = \begin{vmatrix}
x_1 & a_1b_2 & \cdots & a_1b_n \\
a_2b_1 & x_2 & \cdots & a_2b_n \\
\vdots & \vdots & \ddots & \vdots \\
a_nb_1 & a_nb_2 & \cdots & x_n
\end{vmatrix}
\]

\item (10分) 计算行列式 $(n \geq 2)$:
\[
D_n = \begin{vmatrix}
\frac{1}{x_1+y_1} & \frac{1}{x_1+y_2} & \cdots & \frac{1}{x_1+y_n} \\
\frac{1}{x_2+y_1} & \frac{1}{x_2+y_2} & \cdots & \frac{1}{x_2+y_n} \\
\vdots & \vdots & \ddots & \vdots \\
\frac{1}{x_n+y_1} & \frac{1}{x_n+y_2} & \cdots & \frac{1}{x_n+y_n}
\end{vmatrix}
\]

\item (10分) 计算行列式:
\[
D_n = \begin{vmatrix}
1 & 2 & 3 & \cdots & n \\
2 & 3 & 4 & \cdots & 1 \\
3 & 4 & 5 & \cdots & 2 \\
\vdots & \vdots & \vdots & \ddots & \vdots \\
n & 1 & 2 & \cdots & n-1
\end{vmatrix}
\]

\item (15分) 计算行列式:
\[
D_n = \begin{vmatrix}
0 & a & a & \cdots & a \\
b & 0 & a & \cdots & a \\
b & b & 0 & \cdots & a \\
\vdots & \vdots & \vdots & \ddots & \vdots \\
b & b & b & \cdots & 0
\end{vmatrix}
\]

\end{enumerate}

\vfill
\begin{center}
\textit{试卷结束}
\end{center}

\end{document}
